\documentclass[11pt, a4paper, twocolumn]{article}

% -----------------------------------------------------------------------------------------------------
%Pacotes:
\usepackage[brazilian]{babel}
\usepackage[utf8]{inputenc}
\usepackage{graphicx}
\usepackage{multicol}
\usepackage{siunitx}
\usepackage{authblk}
\usepackage[top=2cm, bottom=2cm, left=3cm, right=2cm]{geometry}
\usepackage{listings}
% -----------------------------------------------------------------------------------------------------

\begin{document}
% -----------------------------------------------------------------------------------------------------
\title{Centro Universitário SENAI CIMATEC\\
    \vspace{1cm}
    \textbf{Relatório - PONG \\}
}
\author[1]{Marcelo Borges}
\author[2]{Frederico Barreto}
\author[3]{Júlia Nascimento}
\affil[1]{Engenharia Elétrica - marcelo.borges@aln.senaicimatec.edu.br}
\affil[2]{Engenharia Elétrica - frederico.castellucci@aln.senaicimatec.edu.br}
\affil[3]{Engenharia Elétrica - julia.ribeiro@aln.senaicimatec.edu.br}
\date{Salvador, Bahia\\\today}
% -----------------------------------------------------------------------------------------------------
\maketitle
% -----------------------------------------------------------------------------------------------------    
\begin{abstract} 
    \textbf{O desafio do desenvolvimento do PONG teve como objetivo principal a integração entre hardware e software 
    na contrução de um jogo cujo funcionamento ocorresse de maneira correta e respeitando essa comunicação. As três fases que 
    definiram o escopo dessa prototipagem foram a contrução do circuito físico, conexão entre a IDE do arduino e o Processing, e a 
    fase final de codificação, resultando em um produto que é controlado de modo físicamente entre dois jogadores e é praticado 
    através de uma interface gráfica bem construida. Por fim, concluiu-se que a maior dificuldade ocorreu na codificação de uma nova
    ferramenta e, com isso, necessitando de conhecimento na forma de realizar o versionamento dos respectivos códigos. 
    \\
    Palavras-chave: PONG; Processing; arduino; hardware; software}
\end{abstract}
% -----------------------------------------------------------------------------------------------------
\section{Introdução}
    Para a confecção e idealização do PONG, faz-se necessário primeiramente compreender alguns conceitos por 
    trás das plataformas e equipamentos utilizados na construção do desafio.
    \\
    O \textit{Processing} é uma linguagem de programação de código aberto e ambiente dedesenvolvimento integrado 
    (IDE), construído para as artes eletrônicas e comunidades de projetos visuais com o objetivo de 
    ensinar noções básicas de programação de computador em um contexto visual e para servir como base para 
    cadernos eletrônicos. É considerado um \textit{sketchbook}, uma alternativa de organização de 
    projetos sem ser o um IDE padrão. Ao programar em Processing, todas classes adicionais definidas serão
    tratados como classes internas quando o código é traduzido para Java puro antes de
    compilar. Isso significa que o uso de variáveis e métodos estáticos em classes é proibido
    a menos que você diga que deseja o processamento para o código no modo de Java puro.
    \\
    Já o \textit{Arduino} é uma plataforma de prototipagem eletrônica de hardware livre e de placa única,
    projetada com um microcontrolador Atmel AVR com suporte de entrada/saída embutido, uma
    linguagem de programação padrão que é essencialmente C/C++. 
    Pode ser usado para o desenvolvimento de objetos interativos independentes, ou para ser
    conectado a um computador hospedeiro. Uma placa Arduino é composta por um controlador,
    algumas linhas de E/S digital e analógica, além de uma interface serial ou USB, para
    interligar-se ao hospedeiro, que é usado para programá-la e interagi-la em tempo real.
% -----------------------------------------------------------------------------------------------------
\section{Desenvolvimento}
    O PONG traz como ideia central o mesmo mecanismo de um \textit{Aero Hockey} ou 
    Hockey de mesa, onde existem duas hastes de lados opostos as quais se referem a cada jogador e o objetivo é que cada 
    jogador rebata a bola de modo a evitar que a bola escape e acabe batendo na parede ao invés de na haste. Cada vez que a bola
    bater na parede do adversário (não sendo rebatida por ele), o mesmo pontua. 
    \\
    Para o sucesso e desenvolvimento do desafio, foi necessário utilizar de alguns elementos que conseguissem compôr 
    o objetivo final: um jogo funcional e com uma interface definida. Para isso os elementos utilizados foram:
    \begin{itemize}
        \item Arduino UNO [x1]
        \item \textit{Protoboard} [x1]
        \item Potenciômetros - 10K$\Omega$ [x2]
        \item Botões \textit{Push-buttons} [x2]
        \item Resistores - 10K$\Omega$ [x2]
        \item \textit{Software Processing (sketchbook)}
    \end{itemize}
    Após adquirir os elementos citados, que são requisitos indispensáveis para o sucesso na prototipagem do \textit{game}, foi o 
    momento de estruturar as etapas de construção do jogo. A realização do desafio se baseou em 3 etapas principais: Montagem do 
    circuito físico (Componentes físicos), Integração com o \textit{Processing} e Codificação.
    \\
    A primeira fase foi a de construção física do circuito eletrônico que iria funcionar com o \textit{Hardware}, ou seja,
    a parte onde os jogadores realizam o controle do jogo, fisicamente. Para isso foi necessário utilizar os componentes físicos listados, lembrando que
    essa conexão fisica se dá no arduino e o \textit{Protoboard} é um meio de unir os equipamentos e as entradas (Diditais ou Analógicas) do mesmo.
    Os dois potênciometros funcionam como "controle" de cada jogador, é com ele que os adversários controlam a posição de suas hastes para 
    rebater a bola. O potênciometro possui três terminais e, por isso, sua ligação no arduino foi realizada com um terminal ligado
    ao terra (GND), outro terminal ligado à fase (+5V) e o terceiro terminal ligado à uma entrada analógica. Os \textit{Push-buttons}
    são os botões de ligar e reiniciar o PONG, com isso a ligação desse componente se dá através, também, de três terminais, onde um será conectado à fase 
    (+5V), o segundo terminal é ligado na entrada analógica do arduino, e o terceiro terminal é ligado através de um resisitor de 10K$\Omega$ até o terra (GND).
    Após todos os componentes instalados, foi o momento de alimentar a placa do arduino com uma fonte de 5V, bem como conectar a IDE do mesmo no \textit{Processing}.
    O resultado da montagem do circuíto se apresenta vide figura 1, abaixo.
        \begin{figure}[h]
            \caption{Placa do Arduino conectada}
            \centering
            \includegraphics*[width=7cm]{Imagens/Arduino.jpg}
        \end{figure}
    Na fase dois foi realizada a integração entre o circuito montado e conectado ao arduino e o \textit{software Processing}, através de sua IDE para tranferir as 
    informações necessárias que o \textit{sketchbook} utilza para fazer a modelagem e a integração da parte fisíca à interface gráfica que será criada a partir disso.

      %  \begin{figure}
       %     \caption{IDE + \textit{Processing}}
        %    \centering
         %   \includegraphics*[width=5cm]{Imagens/IDE+PROCESSING.png}
        %\end{figure}
    
    A fase três, que foi a etapa final de desenvolvimento, foi a etapa mais importante e mais complexa, uma vez que ela diz respeito à parte do "cérebro" do protótipo.
    Isso quer dizer que o funcionamento correto do jogo depende diretamente de uma codificação que faça o \textit{hardware} e o \textit{software} se comunicarem corretamente, 
    evitando problemas no decorrer da utilização (jogo). A parte de maior dificuldade por parte do código foi o de definição e funcionamento das angulações que a bola iria 
    se configurar para ser rebatida e ter esse retorno para o outro lado. Ou seja, a etapa de rebatimento da bola e sua angulação coerente e correta para nova direção da bola. 

    
      %  \begin{figure}[h]
      %      \caption{Interface do PONG - \textit{Processing}}
      %      \centering
      %      \includegraphics*[width=5cm]{Imagens/PONG.png}
      %  \end{figure}
    
% ----------------------------------------------------------------------------------------------------   
\section{Conclusão}
Diante do processo de contrução e realização do PONG, foi possível concluir que o desafio proporcionou situações de maior dificuldade na etapa de codificação e de integração
entre a IDE do arduino e o \textit{Prcessing}. Isso se deu por conta, primeiramente, da falta de contato com o \textit{software} e da busca gradativa pela familiaridade com a 
integração de duas partes que precisavam funcionar de modo a quando uma terminava a outra começava. Segundamente, a dificuldade se moldou devido a necessidade de novas funções 
nos códigos, aonde foi necessário uma pesquisa mais afundo que proporcionasse um entendimento suficiente para concluir a atividade proposta.
\\
Também foi muito enriquecedor a aprendizagem no \textit{Processing}, uma vez que o mesmo possibilta a prototipagem de uma interface gráfica para um aplicativo ou site, porém sem 
grandes complexidades e com resultados satisfatórios. Ou seja, o \textit{software} oferece uma integração com IDE's para simulações, testes, ou até mesmo desenvolvimento de produto.
% -----------------------------------------------------------------------------------------------------
\end{document}