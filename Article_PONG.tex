\documentclass[11pt, a4paper, twocolumn]{article}

% -----------------------------------------------------------------------------------------------------
%Pacotes:
\usepackage[brazilian]{babel}
\usepackage[utf8]{inputenc}
\usepackage{graphicx}
\usepackage{multicol}
\usepackage{siunitx}
\usepackage{authblk}
\usepackage[top=2cm, bottom=2cm, left=3cm, right=2cm]{geometry}
\usepackage{listings}
% -----------------------------------------------------------------------------------------------------

\begin{document}
% -----------------------------------------------------------------------------------------------------
\title{Centro Universitário SENAI CIMATEC\\
    \vspace{1cm}
    \textbf{Relatório - PONG \\}
}
\author[1]{Marcelo Borges}
\author[2]{Frederico Barreto}
\author[3]{Júlia Nascimento}
\affil[1]{Engenharia Elétrica - marcelo.borges@aln.senaicimatec.edu.br}
\affil[2]{Engenharia Elétrica - frederico.castellucci@aln.senaicimatec.edu.br}
\affil[3]{Engenharia Elétrica - julia.ribeiro@aln.senaicimatec.edu.br}
\date{Salvador, Bahia\\\today}
% -----------------------------------------------------------------------------------------------------
\maketitle
% -----------------------------------------------------------------------------------------------------    
\begin{abstract} 
\end{abstract}
% -----------------------------------------------------------------------------------------------------
\section{Introdução}
    Para a confecção e idealização do PONG, faz-se necessário primeiramente compreender alguns conceitos por 
    trás das plataformas e equipamentos utilizados na construção do desafio.
    \\
    O \textit{Processing} é uma linguagem de programação de código aberto e ambiente dedesenvolvimento integrado 
    (IDE), construído para as artes eletrônicas e comunidades de projetos visuais com o objetivo de 
    ensinar noções básicas de programação de computador em um contexto visual e para servir como base para 
    cadernos eletrônicos. É considerado um \textit{sketchbook}, uma alternativa de organização de 
    projetos sem ser o um IDE padrão. Ao programar em Processing, todas classes adicionais definidas serão
    tratados como classes internas quando o código é traduzido para Java puro antes de
    compilar. Isso significa que o uso de variáveis e métodos estáticos em classes é proibido
    a menos que você diga que deseja o processamento para o código no modo de Java puro.
    \\
    Já o \textit{Arduino} é uma plataforma de prototipagem eletrônica de hardware livre e de placa única,
    projetada com um microcontrolador Atmel AVR com suporte de entrada/saída embutido, uma
    linguagem de programação padrão que é essencialmente C/C++. 
    Pode ser usado para o desenvolvimento de objetos interativos independentes, ou para ser
    conectado a um computador hospedeiro. Uma placa Arduino é composta por um controlador,
    algumas linhas de E/S digital e analógica, além de uma interface serial ou USB, para
    interligar-se ao hospedeiro, que é usado para programá-la e interagi-la em tempo real.
% -----------------------------------------------------------------------------------------------------
\section{Desenvolvimento}

\begin{figure}[h]
    \caption{Placa do Arduino conectada}
    \centering
    \includegraphics*[width=7cm]{Imagens/Arduino.jpg}
\end{figure}

% -----------------------------------------------------------------------------------------------------   
\section{Conclusão}
% -----------------------------------------------------------------------------------------------------
\end{document}